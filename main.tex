\documentclass[11pt,a4paper,sans]{moderncv}
\usepackage[utf8]{inputenc}
\usepackage{pgfplots}

\moderncvstyle{banking}                            
\moderncvcolor{blue}                               

\usepackage[scale=0.75]{geometry}

\name{Your}{Name}
\title{Your Job Title}
\address{Your Address}{City, State, Zip}
\phone[mobile]{Your Phone Number}                   
\email{Your Email}                                 

\begin{document}

\makecvtitle

\section{Objective}
A brief statement about your career goals.

\section{Education}
\cventry{Year--Year}{Degree}{Institution}{City}{\textit{Grade}}{Description}
\subsection{Semester Scores}
\cvitem{Semester 1}{Score}
\cvitem{Semester 2}{Score}
\cvitem{Semester 3}{Score}
\cvitem{Semester 4}{Score}
\cvitem{Semester 5}{Score}
\cvitem{Semester 6}{Score}
\cvitem{Semester 7}{Score}
\cvitem{Semester 8}{Score}

\section{Skills}
\cvitem{Skill 1}{Description}
\cvitem{Skill 2}{Description}
\cvitem{Skill 3}{Description}

\section{Experience}
\cventry{Year--Year}{Job Title}{Company}{City}{}{General description no longer than 1--2 lines.\newline{}%
Detailed achievements:%
\begin{itemize}%
\item Achievement 1;
\item Achievement 2, with sub-achievements:
  \begin{itemize}%
  \item Sub-achievement (a);
  \item Sub-achievement (b), with sub-sub-achievements (don't do this!);
    \begin{itemize}
    \item Sub-sub-achievement i;
    \item Sub-sub-achievement ii;
    \item Sub-sub-achievement iii;
    \end{itemize}
  \item Sub-achievement (c);
  \end{itemize}
\item Achievement 3.
\end{itemize}}

\section{Projects}
\cvitem{Project 1}{Brief description}
\cvitem{Project 2}{Brief description}

\section{Graph of Semester Scores}
\begin{tikzpicture}
\begin{axis}[
    title={Semester Scores},
    xlabel={Semester},
    ylabel={Score},
    xmin=1, xmax=8,
    ymin=0, ymax=100,
    xtick=data,
    ytick={0,20,40,60,80,100},
    legend pos=north west,
    ymajorgrids=true,
    grid style=dashed,
]
 
\addplot[
    color=blue,
    mark=square,
]
coordinates {
    (1,Score1)(2,Score2)(3,Score3)(4,Score4)(5,Score5)(6,Score6)(7,Score7)(8,Score8)
};
 
\end{axis}
\end{tikzpicture}

\end{document}
